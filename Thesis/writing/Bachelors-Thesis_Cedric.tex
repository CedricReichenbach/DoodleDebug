\documentclass[english]{scrartcl}
\pagenumbering{arabic}
\usepackage[utf8]{inputenc}
\usepackage[english]{babel}
\usepackage{amsmath}
\usepackage{amsfonts}
\usepackage{amssymb}
\usepackage{fancyvrb}
\usepackage{xspace}

% ============================================================
% Markup macros for proof-reading
\usepackage{ifthen}
\usepackage[normalem]{ulem} % for \sout
\usepackage{xcolor}
\newcommand{\ra}{$\rightarrow$}
\newboolean{showedits}
\setboolean{showedits}{false} % toggle to show or hide edits
%\setboolean{showedits}{false} % toggle to show or hide edits
\ifthenelse{\boolean{showedits}}
{
	\newcommand{\ugh}[1]{\textcolor{red}{\uwave{#1}}} % please rephrase
	\newcommand{\ins}[1]{\textcolor{blue}{\uline{#1}}} % please insert
	\newcommand{\del}[1]{\textcolor{red}{\sout{#1}}} % please delete
	\newcommand{\chg}[2]{\textcolor{red}{\sout{#1}}{\ra}\textcolor{blue}{\uline{#2}}} % please change
}{
	\newcommand{\ugh}[1]{#1} % please rephrase
	\newcommand{\ins}[1]{#1} % please insert
	\newcommand{\del}[1]{} % please delete
	\newcommand{\chg}[2]{#2}
}
% ============================================================
% Put edit comments in a really ugly standout display
%\usepackage{ifthen}
\usepackage{amssymb}
\newboolean{showcomments}
\setboolean{showcomments}{true}
%\setboolean{showcomments}{false}
\newcommand{\id}[1]{$-$Id: scgPaper.tex 32478 2010-04-29 09:11:32Z oscar $-$}
\newcommand{\yellowbox}[1]{\fcolorbox{gray}{yellow}{\bfseries\sffamily\scriptsize#1}}
\newcommand{\triangles}[1]{{\sf\small$\blacktriangleright$\textit{#1}$\blacktriangleleft$}}
\ifthenelse{\boolean{showcomments}}
%{\newcommand{\nb}[2]{{\yellowbox{#1}\triangles{#2}}}
{\newcommand{\nbc}[3]{
 {\colorbox{#3}{\bfseries\sffamily\scriptsize\textcolor{white}{#1}}}
 {\textcolor{#3}{\sf\small$\blacktriangleright$\textit{#2}$\blacktriangleleft$}}}
 \newcommand{\version}{\emph{\scriptsize\id}}}
{\newcommand{\nbc}[3]{}
 \renewcommand{\ugh}[1]{#1} % please rephrase
 \renewcommand{\ins}[1]{#1} % please insert
 \renewcommand{\del}[1]{} % please delete
 \renewcommand{\chg}[2]{#2} % please change
 \newcommand{\version}{}}
\newcommand{\nb}[2]{\nbc{#1}{#2}{orange}}
\newcommand{\here}{\yellowbox{$\Rightarrow$ CONTINUE HERE $\Leftarrow$}}

\newcommand\rev[2]{\nb{TODO (rev #1)}{#2}} % reviewer comments
\newcommand\fix[1]{\nb{FIX}{#1}}
\newcommand\todo[1]{\nb{TO DO}{#1}}
\newcommand\meta[1]{\nbc{META}{#1}{purple}}
\newcommand\jr[1]{\nbc{JR}{#1}{orange}}
\newcommand\nes[1]{\nbc{nes}{#1}{blue}}
\newcommand\on[1]{\nbc{ON}{#1}{red}} % add more author macros here
\newcommand\ewe[1]{\nbc{EWE}{#1}{olive}} % add more author

% ============================================================

\newcommand{\DD}{Dood\-le\-De\-bug\xspace}
\newcommand{\Doodle}{\texttt{Doo.\-dle}\xspace}
\newcommand{\println}{\texttt{Sys\-tem.\-out.\-println}\xspace}

% ============================================================

\begin{document}
\title{DoodleDebug}
\subtitle{A shot-gun marriage between System.out.println and object inspectors}
\maketitle

\begin{abstract}
\meta{Only copied for now.}
Developers need effective ways to inspect and explore the run-time state of programs they are developing and debugging.  Modern debuggers and object inspectors are powerful tools, but they can only be used to explore specific points in the execution where breakpoints have been set. As a result, developers often resort to inserting ``print statements'' in code to log the state at multiple points in the execution. Print statements, however are a `poor man's debugger'', since their output is static and cannot be further explored.
We propose to combine the simplicity of print statements with the graphical sophistication and interaction of modern debugging tools.
\DD is a simple API modeled loosely after Java's \println. Objects that are ``printed'' generate graphical views that can be further explored, and can also be used to navigate back to source code in the IDE.
We introduce \DD and present the results of a usability study that shows that \DD can be very effective for common debugging tasks.
\end{abstract}

\section{Introduction}

For understanding and debugging a program, developers rely on tools to track its runtime states during execution.
One method is to insert print statements like \println in Java.
This method is quick and allows its users to compare different states in time of a specific object, represented by a textual output item.
However, this output is static and comes with a couple of conceptual restrictions. On the one hand, the level of detail is hardcoded through the textual representations of objects.
If a developer decides for a simple and clear way of representation, they will need to rewrite their code for any further inspection and re-run the program after every change.
If they initially choose a detailed and verbose object representation, the output will grow and become tedious to read.
Another drawback of textual representation is caused by the simplicity of plain text.
It's one-dimensional and contains no information about graphical features like advanced formatting or colors.

The other half of the two most widely used debugging tools is the family of debuggers.
When utilizing a debugger, a program can be stopped at a specific point of its execution, allowing developers to inspect any detail of this very state.
A clear advantage over textual output is the ability to inspect objects on demand.
Information is only displayed as soon as the user asks for it, nevertheless available without re-running the program.
The drawbacks of debuggers arise from the fact that their inspector is always bound to a specific point in time and therefore makes it impossible to directly compare different states of the same object.

\DD combines the power of the above mentioned tools, erasing the conceptual problems coming with them.
Its output is is generated through an API taking its cue from \println's paradigms.
A developer simply needs to call \Doodle(object) to doodle any object type.
For simple customization of object representations, a class can implement the \texttt{Doodleable} interface which contains 2 methods, \texttt{doodleOn} and \texttt{summarizeOn}.
In contrast to Java's \texttt{toString}, there are two methods, allowing developers to create two levels of detail for representation.
This distinction takes effect in the principle of Semantic Zoom \cite{semantic-zoom}.
Also, they do not generate plain text but draw content on a virtual canvas.

\section{Existing Debugging Tools}
In the world of programming languages, there are two classes of widely used debugging tools:
Textual output like Java's \println and Debuggers.

\subsection{Textual Output}
Java's \println provides a default textual representation for any object, though only primitives and Strings render to a complete (and useful) output in terms of information.
Default renderings are kept lean for other objects, but can be easily improved by overriding \texttt{toString}.
Debugging by directly writing into source code may be easier and quicker than having to equip other debugging tools for a session\ref{some guy}.
Also, objects can be tracked over time, always receiving a copy of it a precisely defined points of execution.

\subsubsection{Best Practice For Textual Output}
For a detailed comprehension of how programmers use textual output for debugging, we wrote a script that searches for such methods in open source project.
It analysed SqueakSource, a hosting service for Smalltalk projects, searching for \texttt{printOn} methods, Smalltalk's equivalent to Java's \println.
\todo{Results}

\subsubsection{Missing Features On Textual Output}
Textual output is static.
As a consequence, users always face a trade-off between detail level and compactness.
If the output misses a bit of information, they will need to go back to the code, edit its rendering and re-run the whole program.

Formatting text-only output is restricted to the usage of text characters, tabs and line breaks.
Also, already printed text or line breaks cannot be reverted.
In particular, nesting complex objects in a consistent way is not possible in a trivial way without breaking the whole pattern of \println. \todo{nice img example here}

\subsection{Debuggers}
Debuggers allow to stop a program's execution at any line of source code.
When stopped, any detail of the current state is inspectable.
Unlike \println, debuggers allow to put additional or remove existing break points when the program is already running, which allows users to change the target piece of code to debug, or to start debugging only after a big amount of time has passed.

\subsubsection{Drawbacks Of Using A Debugger}
Since a debugger only shows the program's state at one point in time, comparison of two time slices is hardly possible.
Important details may only be memorized and mentally compared to the ones at a later point in time.

Eclipse's built-in Java debugger brings no options to customize its output.
When inspecting an object, all its fields are listed by name, coming along with a textual representation.
An object with many for a programmer unimportant fields may suffer the loss of its clarity from that.

\section{Design}
For the design of \DD's output, we consulted literature to carefully plan the different project cycles.
The most important rule we obeyed was to not mix up design and implementation\cite{?}, which could lead to expensive refactorings or remaining design flaws.	

\subsection{The Sketching Cycle}
One simple and powerful way to evaluate drafted user interfaces is to confront future users with them.
We took pen and paper and sketched many possible renderings for data types or other UI elements that came to our mind.
As soon as we had accumulated a good amount of them, we internally discussed them to initially get rid of those that obviously contained usability problems.
The other ones were presented to programmers we called over and they had to look at some pictures and explain what they intuitively believed to see.
\todo{images: sketeches and person staring at them}
If people failed to understand the correct meaning of a sketch, we either improved problematic parts of it or completely refused it.
After improvements or complete re-designs, sketches were presented to other programmers to avoid influencing through the last session.
Sketches that were eventually perceived correctly by most of the people would be accepted for later implementation.

\subsection{Inspecting Doodled Objects}


\section{Implementation}
This section outlines the architecture and mechanisms \DD uses in order to perceive data, process them and display the results.

\subsection{Data Transport}
Since \DD is an Eclipse plugin, it's always running in a different Java VM than the project to be debugged itself.
As a consequence, object data needs to be transported after each \Doodle call.
\DD uses a third-party library\cite{xstream} to serialize doodled objects to XML, then transports the result as string over a connection on localhost using SIMON\cite{simon}.

\subsection{Rendering}
After a request for doodling an object has been received, \DD analyses its type and searches for a fitting rendering in the different customization layers.
If none is available, a default rendering is used.

\subsubsection{Traversing Object Types}
Renderings are iteratively searched for all types and supertypes of an object, starting at the innermost type, defined through the object's class name.
As long as no rendering has been found, the algorithm traverses the inheritance tree in a layer-wise manner, always preferring the class type over interface types inside a layer.
In other words, this algorithm starts searching on the object's direct class and interface types, then goes on for the class' and interface's direct ancestors and repeats until a match was found or all leaves were reached.
The only type excluded from this search is the \texttt{Object} type, since it might be reached before some interface types.

\subsubsection{Output}
As output format, we decided to use HTML.
This enables great formatting possibilities with moderate implementation effort.
Eclipse provides the package \texttt{org.eclipse.swt.browser}, which includes a browser that easily blends into the Eclipse UI.
The rendering used for this browser's content is always the one from the OS's built-in browser and cannot be changed.
On Windows systems, for instace, Eclipse uses the Internet Explorer rendering engine.
As a consequence, we had to be careful when creating HTML output and always test them in different browsers to prevent differences in design on different systems.

\subsection{API}
\DD's API features two API layers for customization.
One layer, the Doodleable interface, utilizes a similar paradigm as Java's \texttt{toString()} method and targets most use cases since it's a simple and quick solution.
For more power and flexibility, users may also provide \texttt{RenderingPlugin}s, which are also used internally for \DD's built-in renderings.
However, \texttt{Doodleable} customizations are always preferred over plugins when both are available for a type.

\subsubsection{The Doodleable Interface}

The \texttt{Doodleable} interface features two methods: \texttt{doodleOn(DoodleCanvas)} for a regular representation and \texttt{summarizeOn(DoodleCanvas)} for a simplified and compact version.
Distinguishing between them enables semantic zoom \cite{semantic-zoom} when inspecting doodles.

\paragraph{DoodleCanvas}
Instead of creating a string like in \println, both methods receive a \texttt{DoodleCanvas} object for drawing contents on it.
The paradigm behind \texttt{DoodleCanvas} adopts the formatting of text in terms of lines and columns.
A virtual cursor starts at the upper-left corner of the canvas.
Drawn objects align one beside each other until a new line is created, which causes the cursor to jump back to the left which one line height offset.
The second formatting option is to create new columns, moving the cursor to a position on top, to the right of the right-most previous object.
\texttt{DoodleCanvas} therefore has three public methods: \texttt{draw(Object)}, \texttt{newLine()} and \texttt{newColumn()}.

\todo{Image here}

\subsubsection{Plugins}
There are cases where implementing the \texttt{Doodleable} canvas is not a satisfying option for developers.
If they don't have access to the source code, there is no (clean) way to add an interface.
Also, the \texttt{Doodleable} API only provides rough formatting options.

\paragraph{Implementing Plugins}
For advanced arrangement or additional features like coloring, \DD includes the option to provide plugins.
They must implement \texttt{RenderingPlugin}, which is most easily done by extending the built-in \texttt{AbstractPlugin}.
Each plugin holds information about the object types it is able to render.
Instead of drawing to a virtual canvas, a plugin receives a html \texttt{Tag} object and renders its own HTML code into this tag.
The principle of semantic zoom is again retained through two different methods for different detail levels.
In addition to HTML code generation, plugins have the option to cleanly provide CSS rules and individually adjust class attributes assigned to object doodles.

\section{User Study}
As we had implemented a new debugging system completely from scratch, we wanted to gather some feedback from potential real world user to verify its usefulness in terms of API and design.
Another goal of such a study would be to reveal problems we might have never found due to our lack of objectivity\cite{the guy that says: "you're not the average user"}.

\subsection{Study Session Setup}
A completely functional release candidate of \DD was used in a fresh Eclipse 4.2 Classic installation, running inside Windows 7 x64 on a ThinkPad T410 with an external mouse.

Since our resources were not unlimited, we decided to do a qualitative study with 7 people as subjects.
We posed 3 different problems to every subject, some of them to be solved with and some without \DD, but with any classical debugging tool they desired.
At the end of the study, we had equally many documentations on problems solved with and without \DD.

Before heading on the study problems, every user had 15-30 minutes to work through a tutorial, play around with \DD in a sandbox and ask questions to the instructor.
When the actual study problems had been revealed, the instructor cancelled their support.

\subsection{Posed Problems}
The 3 different problems aimed on different aspects of debugging in Java.
In the first two problems, the subject was told that a test is failing and they need to find out what the problem is in particular and how to solve it.
For those problems, subjects had full source code access and were allowed to edit anything.

\subsubsection{Sorting}
A couple of gray-scale Color objects are stored in a List.
Some SortingUtil uses a custom comparator with the goal to sort colors by their brightness.
However, sorting the initial color list and comparing it to a hand-built ground-truth list fails.

Solution: The used comparator wrongly treats completely black \texttt{Color} objects as completely white ones.

\subsubsection{Serialization}
A \texttt{SerializingUtil} is used to serialize and de-serialize \texttt{Contact} objects.
Comparison of one particular contact before and after serialization fails.

Solution: The \texttt{long} field \texttt{phoneNumber} in \texttt{Address} as part of \texttt{Contact} changes.
The reason for this is a temporal cast of any \texttt{long} object into an \texttt{int} inside the \texttt{SerializingUtil}.

\subsection{Subjects}
The subjects were convenience-sampled and assured of their anonymity. We informed them about the purpose of our study beforehand, and neither promised nor gave any reward for participating. The participants are enumerated in order of their participation.

\todo{Todo: reduce white space, use a table.}
\begin{description}
% Oskar Truffer
\item [{Alpha}] B.Sc in Mathematics, Minor in computer science. Now master student in computer science. % 60 ECTS
% Remo Diethelm
\item [{Bravo}] B.Sc. in Computer Science. Master Student  in Computer Science.
% Andrei Chis
\item [{Charlie}] M.Sc. in Computer Science. Ph.D. Student in Computer Science.
% Julian Schelker
\item [{Delta}] B.Sc. in Computer Science. Master Student  in Computer Science.
% Raffael Krebs
\item [{Echo}] M.Sc. in Computer Science. B.Sc. Working as Software Engineer, 1 year of industry experience. Experience in Eclipse plugin development
% Roger Kohler
\item [{Foxtrot}] B.Sc. in Computer Science. Master Student  in Computer Science.
% Ueli Scheidegger
\item [{Golf}] Lic.rer.pol. in Economics, Minor Computer Science. Working as Software Engineer, 15 years of industry experience. % 60 ECTS
\end{description}

\subsection{Results}


\section{Future Work}
There are a couple of ideas that might enhance debugging experience and efficiency.
However, they'd need to be tested to verify their usefulness, possibly before implementing them.

\subsection{Highlighting Object Diffs}
When tracking and object over time, the most important information is located where properties of an object have changed.
Therefore, a mechanism comparing objects when they are doodled multiple times could be a powerful feature.

\subsection{Clickable Doodles}
One problem of using \println over a long period of time is, that it's indeed clear which object is printed, but not where the \println class has been made.
To find it, some tedious text search over the whole project is needed in the worst case.
As a solution in \DD, doodles could include a link pointing to the line in source code where the \Doodle call was triggered, similar to the way Throwables are tracked.

\subsection{Debugger Integration}
Since the eclipse debugger only uses textual representations and doesn't allow any customization, a way to enhance it could be to include doodles.
A user would be allowed to switch from the standard textual mode to \DD mode, where an object would be inspected using \DD's rendering.
Or a button beside the textual representation would create a popup with the doodled version of it.

\section{Conclusion}


\section{Acknowledgements}


\end{document}